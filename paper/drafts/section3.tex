
\documentclass[twocolumn]{article}
\begin{document}

%THE THRUST STAND
In our setup we used a thrust stand obtained from RCBenchmark. This thrust stand incorporates an arduino board along with three, three-axis strain guages to measure the forces and the torque derived by the propeller. The board also collects data on the voltage, current and the signal to the ESC's which allows us to measure the power draw of the motor, and the RPM of the motor. Along with this there is a GUI which allows us to collect and visualize the data. From measuring the pulse rate of the board, we can see that the board collects data at a 1kHz speed. At this rate, it is much faster than most ESC's and FC's and therefore is an adequate way to measure how the system responds to external stimuli. 

The wind sensor we have chosen to use is the "Modern Device- Wind Sensor Rev. P". This device is a hot wire anemometer that is designed to measure wind from any direction. Hot wire anemometers work by passing a current through a wire which heats up due to the current. The heat causes the resistance of the wire to change. As wind passes the wire it cools down the wire, which changes the resistance and is measured by the anemometer. Then the anemometer passes more current through the wire to maintain a particular resistance/temperature. This system allows us to measure the wind speed that is resultant of the propeller. We started by measuring the lower and upper limits of the wind sensor. We also incorporated a way to diminish the wind output, which allows us to measure higher speeds.

%Insert a graph here. Explain how the graph shows the lower and upper limits of the wind sensor. 

We also tested how quickly the wind sensor responds to changes in speed. We did this by oscillating the propeller on the thrust stand between high and low speeds. Since the thrust stand measures the force exerted by the propeller, we can sync measuring the force of the propeller on the stand and the wind output with the sensor. This way we can test for the delay in the sensor.
Next we tested how the wind sensor's orientation effects how the sensor measures wind. This is done by maintaining a a thrust


\end{document}                                                                                                                                                                                                                                                                                                                                                                                                                                                                                                                                                                                                                                                                                                                                                                                                                                                                        