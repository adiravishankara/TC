
\documentclass[twocolumn]{article}
\begin{document}

%THE THRUST STAND
In our setup we used a thrust stand obtained from RCBenchmark. This thrust stand incorporates an arduino board along with three, three-axis strain guage to measure the forces and the torque derived by the propeller. The board also collects data on the voltage, current and the signal to the ESC's which allows us to measure the power draw of the motor, and the RPM of the motor. Along with this there is a GUI which allows us to collect and visualize the data. From measuring the pulse rate of the board, we can see that the board collects data at a 1kHz speed. At this rate, it is much faster than most ESC's and FC's and therefore is an adequate way to measure how the system responds to external stimuli. The measurement from the thrust stand will be our "ground truth".

The wind sensor we have chosen to use is the "Modern Device- Wind Sensor Rev. P". This device is a hot wire anemometer that is designed to measure wind from any direction. Hot wire anemometers work by passing a current through a wire which heats up due to the current. The heat causes the resistance of the wire to change. As wind passes the wire it cools down the wire, which changes the resistance and is measured by the anemometer. Then the anemometer passes more current through the wire to maintain a particular resistance/temperature. This system allows us to measure the wind speed that is resultant of the propeller. From measuring the data rate of the sensor, we measure that the sensor communicates and measures at a rate of 4kHz. We started by measuring the lower and upper limits of the wind sensor. We also incorporated a way to diminish the wind output, which allows us to measure somewhat higher speeds. The test shows us that the minimum value with $0 km/h$ is averaged at $450 mV$, while a speed of $15 km/h$ results in a voltage of $750 mV$.

%Insert a graph here. Explain how the graph shows the lower and upper limits of the wind sensor. 

We also tested how quickly the wind sensor responds to changes in speed. We did this by oscillating the propeller on the thrust stand between high and low speeds. Since the thrust stand measures the force exerted by the propeller, we can sync measuring the force of the propeller on the stand and the wind output with the sensor. This way we can test for the delay in the sensor. The delay that the wind sensor faces is approximately $4ms$ in relation to the thrust-stand. While this is a 

%Insert graph on oscillating fan measurement

Next we tested how the wind sensor's orientation and placement effects how the sensor measures wind. This is done by maintaining a stable thrust on the thrust stand. Next we position the wind sensor in various locations moving in a grid like pattern. With this method we can determine areas of most wind measurement and least. Next we move the sensor from a distance, towards the propeller and measure the average voltage. The area with the highest voltage results in the highest sensing. Lastly we tested how the wind sensor measures changes in the wind in different orientations; specifically normal to the wind, parallel to the wind, and away from the wind. 

%Insert the three part graph here with wind placement data.

From the collected data, we see that there is approximately a semicircle around which the wind sensor can be placed, and a distance of 15 cm away from the propeller is the convergence point 


\end{document}                                                                                                                                                                                                                                                                                                                                                                                                                                                                                                                                                                                                                                                                                                                                                                                                                                                                        