\documentclass{article}
\begin{document}


\section{I}

%How do drones work, the control system
Multirotor Aerial Vehicles (MAV), specifically quadrotors, have become quite commonplace in many industries ranging from filmography to racing to even military operations. Over the past decade, there has been an incredible amount of research and advancement within this field leading to its proliferation and allowing for the technology to become more accessible. One of the major areas of research regarding MAV's is the control system that directs the motor-rotor system. The control mechanism can be as simple as a motor-rotor control, all the way to controlling the torque, or thrust or one of many other variables. The control of the motor-rotor has to account for multiple different factors, from the ambient wind speed, the rotation, the pitch of the propellor, all the way to the high level control system of the ESC. \\

In general, there are two different types of control loops; open and closed. Open control loops are what the majority of MAV control systems are. There is a flight controller, that sends a signal to the ESC that applies a particular voltage to the motor based on previously calibrated settings. In this method, the final result of the motor spinning is unknown to the flight controller. Therefore given any disturbances, there will be large variances in the response of the overall system. In response to these systems there are smarter, closed loop control. This is the type of control we will focus on. Closed loop systems provide a certain way to meaure the output of the rotor based on the input from the FC. This allows for a smarter, more robust system and greater control of the output. 

%What are some of the technologies available, what are their pros and cons

%Intro into the problem

%Intro into wind sensor

%INtro into the mahoney Paper

\section{II}
%Theory of the mahoney paper

%Theory of the wind sensor

%How they will be used together

\section{III}
%Proof of the mahoney paper

%Proof of the wind sensor

\section{IV}
%Experimental setup

%Data

%Results

\section{V}
%Conclusion

%References
\end{document}
