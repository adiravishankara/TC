\documentclass{article}
\begin{document}


\section{Introduction}


Control systems on multirotor aerial vehicles have had many different iterations, each of which have added a significant level of control to the system. The control mechanism can be as simple as a motor-rotor control, all the way to controlling the torque, or thrust or one of many other variables. The control of a multirotor propellor has to account for multiple different factors, from the ambient wind speed, the rotation, the pitch of the propellor, all the way to the high level control system of the ESC. \\

General control systems are what are called open loop control systems. The Flight controller sends a value that the propellor should spin at, the ESC converts that value into a form that the motor can handle and then the motor spins. From this system there is no feedback as to what the motor is doing[1]. You wouldn't know if the motor was spinning, or if there was a propellor on the motor or one of many other faults. One way that this problem was approached is that the voltage and current draw of the motor is measured and from there a calibrated system would be able to determine how fast, how much torque and how efficiently a motor is spinning[1]. These are classified as closed system approaches[3]. Another method used is a torque sensor (strain guage) attached either to the frame or to the motor[21]. This way we can see the the force the motor exerts and therefore determine hwo much external forces are acting on the drone so that the fc can respond accordingly[21]. Another method used is a pitot tube to measure the wind speed of either the external wind or the output of the propellor[18]. This way we can control the output based on the wind speed from the propellor which is an analog for the thrust output[18]. Lastly what some people use is the IMU's in the drone to measure changes in the drone[7]. There is an expected speed and direction if that changes we know that there are external forces added. While these are are extremely clever and efficient methods they do have their own downfalls. Just measuring the current and voltage would lead to lots of noise errors, and also it does not respond to sudden changes in wind as quick as we need it. The torque sensor is subject to lots of errors from noise but also has a low response rate and therefore quick changes would not be possible. The pitot tube method is similar to what we are proposing but rather than a pitot tube we are using a hot wire anemometer. The pitot tube is not as sensitive to changes as we need for this application. Lastly the IMU's, while they provide the reaction of the entire drone, they do not accurately show how each propellor is reacting. It requires intense computation to produce a model each time we sense a change. 


Our paper will compare our system with those of existing systems. One last method of measurement to test against is the mahoney paper. This paper will be what our experiment will hopefully prove better. The mahoney paper uses a thrust estimation based on current, voltage and measurement of several properties of the motor and propellor. This data is used within an algorithm that allows the drone to estimate an appropriate response to maintain a constant thrust. 

Our system, a control system with a thrust sensor which provides a feed back loop for the system to check if the appropriate thrust is being provided by the propellors. 

\section{Theory}

Since we are just focusing on on the static test, there are some assumptions we are making and some math we are skipping. 

We are following the math along with the Mahoney Paper.

We have to start by defining the three different velocities that we are measuring.\\

$v_{x} , v_{y} , v_{z}$  Is the velocity of the drone, the body frame velocity. \\
$v^s_{x} , v^s_{y} , v^s_{z}$ is the velocity of the wind approaching the propellors, the stream velocity. \\
$v^i_{x} , v^i_{y} , v^i_{z}$ is the induced velocity of the propellors.\\

Since we are doing only the static tests, we do not have to worry about horizontal winds, so we can say $v_s = v^s_{z}$ and $v_i = v^i_{z}$
What the wind sensor will notice is $v_{tot} = v_{i} + v_{s}$\\

Following along with the mahoney paper we have
\begin{equation}
\lambda = \frac{v_{i} + v_{s}}{\varpi R}
= \frac{v_{i}}{\varpi R}+\frac{v_{s}}{\varpi R}
=\lambda_{i} + \lambda_{s}
\end{equation}

Next we go on to calculate the Thurst relationship with the voltage and current through the equations F1Td and F2Td.

\begin{equation}
T_{d} = \frac{C_{T}}{K^2_{e}}(v_{a} - i_{a}R_{a})
\end{equation}

$K_{e}$ in this case is obtained by calibrating the system and applying a linear regression while $v_{a}, i_{a}, R_{a}$ are all obtained from the power sent to the motor. $C_{T}$ can be calculated by 
\begin{equation}
C_{T} = c_{p} \lambda ^i \lambda 
\end{equation}
\begin{equation}
c_{p} = 2 \rho A R^2 \nonumber
\end{equation}
\begin{equation}
C_{T} = 2 \rho A R^2 \frac{v_{i}}{\varpi R}\frac{v_{i} + v{s}}{\varpi R}  = \frac{2 \rho A (v_{i}^2 + v_{i}v_{s})}{\varpi ^2}
\end{equation}

After calculating these variables we can start to work towards how we control the motor. From the mahoney paper, they individually calculate the appropriate voltage and current to send to the motor. From teh paper to calculate $f_{1}(T_{d})$
\begin{equation}
T_{d} = \frac{C_{T}(v_{a} - i_{a}R_{a})^2}{k_{e}^2}
 = \frac{2 \rho A}{\varpi ^2 k^2_{e}}(v^2_{i} + v_{i}v_{s})(v_{a} - i_{a}R_{a})^2
\end{equation}
$f_{1}(T_{d})$ is used to calculate the voltage. This can be rearranged to obtain $v_{a}$
\begin{equation}
v_{a} = k_{e} \varpi \sqrt{\frac{T_{d}}{2 \rho A (v_{i}^2 + v_{i}v_{s})}} + i_{a}R_{a}
\end{equation}
similarly from the paper we can calculate F2Td where 
\begin{equation}
T_{d} = C_{T} \frac{k_{q}}{C_{Q}}i_{a}
\end{equation}
from applying a similar algebraic reorganization, we get 
\begin{equation}
T_{d} = 2 \rho A \frac{v_{i}^2 + v_{i}v_{s}}{\tau} k_{q}i_a
\end{equation}
from which we can solve for the current and we get
\begin{equation}
i_a = \frac{T_d \tau}{2 \rho A (v_i^2 + v_i v_s) k_q}
\end{equation}
Once these values are calculated, they are sent to the ESC at which point the esc controls the speed of the motor. 
Data from the ESC is collected to help make the thrust estimation. 
Our wind sensor data will be replacing the estimation for the real value which will allow for a more accurate and crisp response.
\end{document}

 

